\renewcommand{\chaptertitle}{Homework}
\setdate{Sepember 7, 2025}

\section{\chaptertitle}

During the summer of 2025, the Kyiv School of Economics (KSE) brought together an eclectic group of people to embark on an ambitious project: redesigning engineering education in Ukraine.

You might ask, \textit{why}? The Ukrainian education system, inherited from the Soviet Union, still carries many of its shortcomings: rigidity, obsolescence, and very limited room for change, to name just a few.

Although the Soviet Union produced tremendous numbers of engineering graduates each year, after its collapse the majority of them ended up unemployed. These facts suggest that:

\textbf{1.} the demand for engineers had been artificially inflated by those in power;

\textbf{2.} the quality of their education was not particularly high.

Taking this into account -- along with the more recent experiences of those who have studied engineering in Ukrainian universities—it becomes clear that reform is long overdue.

To achieve that end, we are drawing inspiration from what Olin College did some 30 years ago. Yet instead of simply copying their model, we have chosen to start almost from scratch, while still building on the valuable lessons Olin has accumulated over the years.