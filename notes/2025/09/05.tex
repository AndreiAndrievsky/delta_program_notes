\renewcommand{\chaptertitle}{Research}
\setdate{September 5, 2025}

\section{\chaptertitle}

\subsection{Agenda}
% $\textbf{Agenda}$:

9:30-10:00 - Coffee Tea Talk (please someone bring some tea bags)

10:00-10:30 - Updates $\&$ Intro

10:30-11:30 - Intro to Human-Centered Design

11:30 - 12:00 transfer GAPA (alumni profile) to the action plan and timeline

12:00-13:00 - Part 1 Teams Activity - Research plan (research objective=>research questions)

13:00-14:00 - Lunch

14:00-15:00 - Part 2 Team Activity - Research plan (research questions=>target audience=>methods)

15:00-15:30 - Schedule of the next week 

15:30-17:00 - Groups work - plan for next week research


\subsection{Naming}

Delta ($\Delta$)

I like it and also it opens cool possibilities for branding

\subsection{Homework}

Prepare a 1-pager describing our project for Olin faculty members

$\textbf{Due}$: early (?) next week

\subsection{Design thinking}

\begin{notebox}
    Ishikawa diagrams???
\end{notebox}

% same for theorem/example
\begin{definition}{Human-centered design}{}
A process that helps us improving products/policies/etc. It's iterative. We iterate -- don't strive for perfection from the very beginning.
\end{definition}



Here's the framework of using it:


\begin{tcolorbox}[enhanced,colback=Lavender!20,colframe=Purple,title=Design Thinking Framework]
\textbf{1. Explore:}
\begin{itemize}
    \item Go to target audience
    \item Talk, observe, read other researches
\end{itemize}

\textbf{2. Define:}
\begin{itemize}
    \item What problem we should solve
\end{itemize}

\textbf{3. Ideation:}
\begin{itemize}
    \item Generate as many ideas as you can
    \item Prioritize
\end{itemize}

\textbf{4. Prototype:}
\begin{itemize}
    \item Self-explanatory, do it quick
    \item Must be shitty, do one, then do the next, don't stick to your guns
\end{itemize}

\textbf{5. Test:}
\begin{itemize}
    \item Test prototypes, part of the project
    \item Figure out is there is a desirability from the audience
\end{itemize}

\textbf{6. Repeat:}
\begin{itemize}
    \item Go to stage 3, Ideation
\end{itemize}

\textbf{7. Implement:}
\begin{itemize}
    \item When previous steps are done, invest in implementation
\end{itemize}

\end{tcolorbox}

\subsection{Transfer to GAPA}

\begin{warningbox}
    When we do the actual research???
\end{warningbox}

- Preparation for the research\dots

- Data collection

- Data analysis (probably properly done after Olin)

- Alumni profile (maybe not a full one)

- Research report draft (less important than profile)

--- Olin here ---

- Maybe revise the alumni profile after the trip

- Research report


For edebo:

- Initial program (1 year, maybe?? or full 4)

- 3 core faculty, professional, each no more than 5 courses a year

- a full 4 year curriculum only disciplines, and teachers

- teching results

- curriculum for a year with credits


\begin{notebox}
    One pager about our values and what we want to incorporate into our $\Delta$-project

    By the end we should have 3 papers:

    - public

    - internal (agreement)

    - manifesto (bridges previous two)
\end{notebox}

- prepare for Design nature

\begin{tipbox}
    Work with Timur secretely to design an engineering challenge for the faculty team
\end{tipbox}


\subsection{Research methods}

Epistemology??

Positivity and Interpretabiliry (knowledge)

Quantitative and qualitative




