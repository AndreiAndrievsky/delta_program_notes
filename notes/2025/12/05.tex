\startchapter{Week closer}{Dec 5, 2025}


\textbf{Agenda}

13:00–13:10 Check-in\\
13:10–13:20 Updates from Rebecca and Olga\\
13:20–14:30 Updates from working groups\\
14:30–14:45 Coffee break\\
14:45–15:30 Experiments: progress, bottlenecks + plans for next week\\
15:30–16:00 Finalizing the program naming\\
16:00–16:10 Plans for next week\\
16:10–16:20 Plus/Delta of the session\\
16:20–16:30 Check-out

\subsection{Check-in}

I came with Andrii and we were a bit late. Nastasia told
me they were doing a check-in and the task was to share
one sentence describing our summary of the week. The
remaining time was just 1m20s.

I spent the best part of the check-in in an internal 
struggle whether I should note each person's takeout and 
soon about a half of people have alreaty talked and I ended
up not noting anything.

In the middle of Kate's speech Rebbecca walked in a woman in
a green sweater. No idea who she is but she has a "KSE team"
bage.

\begin{quotebox}[Oleg]
Be resilient.
\end{quotebox}

Just learned that the woman's name is Monica.

\subsection{Updates from Rebecca and Olga}

We are working monday and friday in the office together.
Other days are for group work. If we need something like
we can't be present or need some leave/vacation time we 
should reach out to Nastasia. Also Rebbecca wants to have
one-to-ones with everyone in the team to know how everyone
feels.

There will be a final meeting this year on Dec \nth{22} to 
wrap up and finish the sprint 3. The idea is to meet 
somewhere not in the office so it feels more like a retreat.

Taya's erstwhile teacher from her school used to do 
workshops in her school and agreed to do some with us.
He plans to be here on Tuesday next week.

One speciality is applied is mechanics, the second adds 
robotics, and on top of that the third one has electronics.
We need to finilaze the name of the program so Slava can
submit all the paperwork. It was planned that each working
group has 10 min. But in the middle of Olga's speech 
Rebbecca tampered with the timer and reset it to 10 min.

Rebbecca and Nastasia talked to a new marketer who develops
KSE's new brand/identity. KSE will be going through a 
rebrending, it will happen this summer which coincides with 
the start of our program. KSE not is very polished, looks
like a business school and they want to change it to 
something more hands-on and approachable. KSE's motto:
\texttt{Guided by science and driven by defience}.

Recruitment and addmission worked on personas of our 
prospective students. They are:
\begin{itemize}
    \item The pragmatist
    \item The community seeker
    \item The ambitious strategist
\end{itemize}

It's not like every person is one of three. They are like
basis vectors and every person is a linear combination of 
all three.

KSE HR team is already working on finding new people for our
team. We are working on it as well. We took a list of 
winners of KSE's charitable foundation competition. Andrii 
and Roma are working on that.

\begin{quotebox}[Nastasia and Kate]
Every discussion is a rabbit hole.
\end{quotebox}

\subsection{Updates from working groups}

During our last science\&math and engineering meeting we\dots
decided that we need another one. Nadia sent us a doodle link
so we can select the time for the next one. We should be
prepared for it, to go through the miro boards and come up 
with ideas how to combine ... \textit{don't remember the end
of the sentence}.

We went through what MnS and engineering put into each 
term and aligned them. The ultimate idea is to make a list
of common learning objectives.

How can Olin support us and how we can work closer with KSE.
Monica itrodused herself, experimental economist. She likes
experiments and experimenting. She talked to Rebbecca about
her idea. She was thinking about the minesweeper game and 
the situation in Ukraine. The idea is to make a game that 
would bring some money to the field of demining (which will
still be done by experts).

From this we moved to each person's chunk experiment they 
prepare. Andrii filled Monica in on what we are doing and 
what the experiments are about. I didn't note on those
because I was a little exhausted and it was a lot. I liked
Dasha's idea very much, it's about teaching astro-physics
with a jupyter notebook with python simulations. Other 
people's ideas were just as valuable, but this is my 
personal top.

\begin{quotebox}[Dasha]
I would never apply to this shit.
\end{quotebox}

