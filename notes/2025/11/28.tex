\startchapter{Sync week closer}{Nov 28, 2025}

\textbf{Agenda}:

10:00-10:30 Check-in.\\
10:30-10:45 Re-anchor to Module goals\\
10:45-11:00 Talk about definition of ready \& done, acceptance criteria\\
11:00-11:20 Co-create Sprint goal\\
11:20-11:30 Break\\
11:30-11:40 Visualize team capacity (confirm availability per person)\\
11:40-12:15 Review backlog of tasks writing them into stickers with definition done. (Retro items, teams plans, other tasks)\\
12:15-12:30 Prioritize backlog based on Sprint goal.\\
12:30-13:30 Lunch\\
13:30-14:00 List top 3 deliverable for Sprint 2 with definition of done and responsible.\\
14:00-14:30 List secondary tasks to complete and responsible\\
14:30-15:00 Agree on groups structure based on capacity, retro and plans.\\
15:00-15:15 Break\\
15:15-15:45 Summarize plan\\
15:45-16:15 Reflection of the day.\\
16:15:16:30 Checkout.

\par\noindent\rule{\textwidth}{0.4pt}
\par\noindent

We started with blue-sky goals. Most people were opposed
to it and mentioned that "simply building a program" isn't
a thing and it's difficult. Nastasia mentioned that this 
(third) sprint we still diverge (diverge-converge approach)
and this is our priority because this is aligned with the
design thinking. Otherwise we will create an ordinary 
program that will be ten a penny.

Then we discussed the session opening (the check-in) and
why it's important (team dynamics/cohesion) and today's
one is: to articulate one thing that we think is going to 
bring our goal to fruition. Program is a list of 
specializations and courses for each year. We need a 
program to figure out what faculty members we lack.

Roma talked about scrum methodology, definition of done, 
and definition of ready.

\begin{quotebox}[Andrii]
Why the heck do we need Rebbecca and Olga?
\end{quotebox}

After this we started writing down our backlog:
\begin{itemize}
    \item faculty gaps
    \item decision on majors
    \item[$\heartsuit$ $\bigstar$] sync with teams $\bigstar$ $\heartsuit$
    \item preparation for Open Day 
    \item makerspace in the office
\end{itemize}

\textbf{To-do list sprint 2}

\begin{itemize}
    \item chunk experimentation (creation, test)
    \item edujam
    \item open day presentation
    \item brand
    \item EDEBO registration
\end{itemize}

When developing a course the author develops a detailed 
break down, then experiments with students, improves,
iterates. At the end of this sprint we present this 
"experementator diaries" for each learning chunk at the end
of this sprint. 

We had quite some disscussions about various things, but I
didn't note them. Rebbecca and Olga announced some 
changes in out team's composition. Then we filled them in on
what we've discussed today.